\chapter{はじめに}
\label{chap:hajimeni}

\section*{この本の流れ}
\addcontentsline{toc}{section}{この本の流れ}
\label{sec:-1}

まず、第1章は本書で作った qiita2reviewの使い方、第2章はQiita上で数式を含んだ記事の書き方、第3章、第4章は組版システムRe:VIEWを使った例、第5章、第6章はQiitaとの連携を記しています。

\section*{Web連携}
\addcontentsline{toc}{section}{Web連携}
\label{sec:-2}

この記事はWebでも読めます。Qiitaで公開していますので印刷が見づらいところは参照して下さい。

\section*{コードの公開について}
\addcontentsline{toc}{section}{コードの公開について}
\label{sec:-3}

公開まで手が廻りませんでした。落ち着いたらPublic Domainとして公開しますので twitter @nanbuwks で検索するか、お問い合わせください。

\section*{お詫び}
\addcontentsline{toc}{section}{お詫び}
\label{sec:-4}

システムを作るのが長引いて、書式を調整する時間がありませんでした。
コードがは見えたりしてますが、Webで補完ください。

\begin{reviewimage}
\includegraphics[width=\maxwidth]{./images/88x31.png}
\label{image:hajimeni:88x31}
\end{reviewimage}

2017/04/10 03:31
